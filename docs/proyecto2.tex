\documentclass{article}

% Language setting
% Replace `english' with e.g. `spanish' to change the document language
\usepackage[spanish]{babel}

% Set page size and margins
% Replace `letterpaper' with`a4paper' for UK/EU standard size
\usepackage[letterpaper,top=2cm,bottom=2cm,left=3cm,right=3cm,marginparwidth=1.75cm]{geometry}

% Useful packages
\usepackage{amsmath}
\usepackage{graphicx}
\usepackage[colorlinks=true, allcolors=blue]{hyperref}

\title{Proyecto 2. Esteganografía por el método LSB}
\author{Pérez Romero Natalia Abigail}

\begin{document}
\maketitle

\section{Introduccion}

\subsection{Plan de trabajo}

Ya que lo que recibe es un texto y una imagen, y el texto debe ser ocultado en la imagen se necesita:

\begin{enumerate}
    \item Un método que convierta el texto en binario (sería un plus cifrarlo primero, ocultar el texto cifrado)
    \item Un método que convierta la imagen en binario
    \item Un método que modifique el último bit
    de la imagen
    \item método que lea texto de archivo
    \item Método que obtenga el mensaje una imagen
\end{enumerate}

\subsection{Pruebas}
\begin{enumerate}
    \item Prueba del método que pasa a binario a texto
    \item Prueba del método que pasa de texto a binario
\end{enumerate}

Para las pruebas tengo dos ideas:
\begin{enumerate}
	\item El abecedario
	\item Un motón de cadenas aleatorias
\end{enumerate}


\end{document}